\section{Development of new algorithms \label{sec:dvelopments}}

\subsection{Improved jet-track association}

\subsection{Combined secondary vertex algorithm}

The discrimination power of all secondary vertex $b$-taggers described
in \cite{btaggingPAS2009} can be further improved by combining them in
a multivariate analysis (MVA) \cite{Weiser:2006md}.  A new
implementation of this technique is in development, using a general
MVA framework in CMSSW.

A simple classifier was chosen as a first step, to make sure that the
workflow is well-understood.  Input variables are combined using a Naive
Bayes likelihood ratio (which assumes that they are uncorrelated), and
are normalized and rotated to remove linear correlations.

The discriminator is separately optimized in three broad classes: RecoVertex (a
secondary vertex is fully reconstructed), PseudoVertex (no secondary
vertex is reconstructed, but at least two tracks are inconsistent with
the primary vertex with a transverse impact parameter significance
greater than 2.0), and NoVertex (neither of the above).  These classes
determine the set of input variables available for optimization.

The work on this project is ongoing.  The MVA framework is modular
enough to replace classification engines with minimal impact on the
workflow; thus, the project has a natural upgrade path to more
sophisticated MVA techniques.
