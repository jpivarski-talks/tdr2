\section{Introduction}
The identification of jets originating from b quarks  is a crucial
element for many physics analyses. In particular, high branching ratios to b quarks characterize a variety
of Standard Model (SM) and discovery channels like the measurement of bottom or top pair
production, the search for Higgs bosons and different other New Physics scenarios. 

The hard fragmentation, the long lifetimes and high masses of B hadrons and the relatively
high fraction of semileptonic decays distinguish these jets from those originating from gluons,
light quarks and - to a lesser extent - from c quarks. Due to its precise inner tracking system
and its lepton identification capabilities the CMS experiment is particularly suited for exploiting
these features.

Detailed definitions of the studied quantities are given in
\cite{btaggingPAS2009}, along with an explanation of the b-tagging
algorithms. The b-tag commissioning  with first collision data at 7~TeV is reported in
\cite{btaggingPAS2010}. The present note presents an update of the
commissioning activities including data from the 2011 run. The
validation of the main variables is discussed in Sections \ref{sec:trackselection} to \ref{sec:muonjets}. One
of the main differences to \cite{btaggingPAS2010} is the presence of
pileup, which is discussed in Section \ref{sec:pileup}. The development and
commissioning of additional higher-level b-tagging algorithms is
documented in Section~\ref{sec:dvelopments}.  