%%%%%%%%%%%%%%%%%%%%%%%%%%%%%%%%%%%%%%%%%%%%%%%%%%%%%%%%%%%%%%%%%%%%
%
%   Style for CMS Computing / Physics Technical Design Reports
%
%   Lucas Taylor  4 Feb 2005,   Revised  12 Oct 2005
%
%%%%%%%%%%%%%%%%%%%%%%%%%%%%%%%%%%%%%%%%%%%%%%%%%%%%%%%%%%%%%%%%%%%%

%  the following line is edited by the tdr script to change or to pass
%  additional options:
\documentclass[11pt,twoside,a4paper,note]{cms-tdr}
\def\svnVersion{29932:59897MP}

%%%%%%%%%%%%%%%%%%%%%%%%%%%%%%%%%%%%%%%%%%%%%%%%%%%%%%%%%%%%%%%%%%%%

\begin{document}
%%%%%%%%%%%%%%%%%%%%%%%%%%%%%%%%%%%%%%%%%%%%%%%%%%%%%%%%%%%%%%%%%%%%
%
%  Common definitions
%
%  N.B. use of \providecommand rather than \newcommand means
%       that a definition is ignored if already specified
%
%                                              L. Taylor 18 Feb 2005
%%%%%%%%%%%%%%%%%%%%%%%%%%%%%%%%%%%%%%%%%%%%%%%%%%%%%%%%%%%%%%%%%%%%


%%%%%%%%%%%%%%%%%%%%%%%%%%%%%%%%%%%%%%%%%%%%%%%%%%%%%%%%%%%%%%%%%%%%
%
% Hyphenations (only need to add here if you get a nasty word break)
%
\hyphenation{env-iron-men-tal}%    just an example
\hyphenation{had-ron-i-za-tion}
\hyphenation{cal-or-i-me-ter}
\hyphenation{de-vices}
%
% Hyphenations-end
%% Customizable fields and text areas start with % >> below.
% Lines starting with the comment character (%) are normally removed before release outside the collaboration, but not those comments ending lines

% svn info. These are modified by svn at checkout time.
% The last version of these macros found before the maketitle will be the one on the front page,
% so only the main file is tracked.
% Do not edit by hand!
\RCS$Revision: 57905 $
\RCS$HeadURL: svn+ssh://pivarski@svn.cern.ch/reps/tdr2/notes/AN-11-180/trunk/AN-11-180.tex $
\RCS$Id: AN-11-180.tex 57905 2011-05-27 08:42:28Z alschmid $
%%%%%%%%%%%%% ptdr definitions %%%%%%%%%%%%%%%%%%%%%
%\input{ptdr-definitions} %These have been replaced by the equivalent style file
%%%%%%%%%%%%%%%  Title page %%%%%%%%%%%%%%%%%%%%%%%%
\cmsNoteHeader{AN-11-180} % This is over-written in the CMS environment: useful as preprint no. for export versions
% >> Title: please make sure that the non-TeX equivalent is in PDFTitle below
\title{Status of b-tagging tools for 2011 data analysis}

% >> Authors
%Author is always "The CMS Collaboration" for PAS and papers, so author, etc, below will be ignored in those cases
%For multiple affiliations, create an address entry for the combination
\address[neu]{University on the Moon}
\author[cern]{The CMS Collaboration}

% >> Date
% The date is in yyyy/mm/dd format. Today has been
% redefined to match, but if the date needs to be fixed, please write it in this fashion.
% For papers and PAS, \today is taken as the date the head file (this one) was last modified according to svn: see the RCS Id string above.
% For the final version it is best to "touch" the head file to make sure it has the latest date.
\date{\today}

% >> Abstract
% Abstract processing:
% 1. **DO NOT use \include or \input** to include the abstract: our abstract extractor will not search through other files than this one.
% 2. **DO NOT use %**                  to comment out sections of the abstract: the extractor will still grab those lines (and they won't be comments any longer!).
% 3. **DO NOT use tex macros**         in the abstract: External TeX parsers used on the abstract don't understand them.
\abstract{
The identification of jets containing the weak decay of a B-hadron is
an essential tool for a wide range of analyses in the context of the
Standard Model and beyond. A variety of algorithms exploit the long
lifetime and the presence of soft leptons to discriminate these jets
from those associated to light quarks. The status of the b-tagging
tools and their commissioning with 2011 data is presented. New
developments and improvements of the b-tagging algorithms are also
documented.  
}

% >> PDF Metadata
% Do not comment out the following hypersetup lines (metadata). They will disappear in NODRAFT mode and are needed by CDS.
% Also: make sure that the values of the metadata items are sensible. For APS submissions, they are automatically converted to APS keywords.
\hypersetup{%
pdfauthor={Alexander Schmidt},%
pdftitle={Status of b-tagging tools for 2011 data analysis},%
pdfsubject={CMS},%
pdfkeywords={CMS, BTV, physics, software}}

\maketitle %maketitle comes after all the front information has been supplied

% >> Text
%%%%%%%%%%%%%%%%%%%%%%%%%%%%%%%%  Begin text %%%%%%%%%%%%%%%%%%%%%%%%%%%%%
%% **DO NOT REMOVE THE BIBLIOGRAPHY** which is located before the appendix.
%% You can take the text between here and the bibiliography as an example which you should replace with the actual text of your document.
%% If you include other TeX files, be sure to use "\input{filename}" rather than "\input filename".
%% The latter works for you, but our parser looks for the braces and will break when uploading the document.
%%%%%%%%%%%%%%%
\section{Introduction}
The identification of jets originating from b quarks  is a crucial
element for many physics analyses. In particular, high branching ratios to b quarks characterize a variety
of Standard Model (SM) and discovery channels like the measurement of bottom or top pair
production, the search for Higgs bosons and different other New Physics scenarios. 

The hard fragmentation, the long lifetimes and high masses of B hadrons and the relatively
high fraction of semileptonic decays distinguish these jets from those originating from gluons,
light quarks and - to a lesser extent - from c quarks. Due to its precise inner tracking system
and its lepton identification capabilities the CMS experiment is particularly suited for exploiting
these features.

Detailed definitions of the studied quantities are given in
\cite{btaggingPAS2009}, along with an explanation of the b-tagging
algorithms. The b-tag commissioning  with first collision data at 7~TeV is reported in
\cite{btaggingPAS2010}. The present note presents an update of the
commissioning activities including data from the 2011 run. The
validation of the main variables is discussed in Sections \ref{sec:trackselection} to \ref{sec:muonjets}. One
of the main differences to \cite{btaggingPAS2010} is the presence of
pileup, which is discussed in Section \ref{sec:pileup}. The development and
commissioning of additional higher-level b-tagging algorithms is
documented in Section~\ref{sec:dvelopments}.  
\input{HighLevelTrigger.tex}
\input{samplesandselection.tex}
\section{Track selection \label{sec:trackselection}}
Tracks are associated to jets using a cone in $\eta - \phi$ space defined as $\Delta R = \sqrt{\Delta \eta^2 + \Delta \phi^2} < 0.5$. A new development which introduces a variable cone size is discussed in Section~\ref{sec:dvelopments}. 

Furhter selection criteria for tracks are listed in the following. The corresponding distributions for all tracks within a distance $\Delta R < 0.5$ to the jet axis are displayed in Figures~\ref{fig:inputVars1} to \ref{fig:inputVars3}.
\begin{itemize}
\item number of pixel hits $>= 2$  (Figure \ref{fig:inputVars1})
\item number of tracker hits (including pixel) $>= 8$ (Figure \ref{fig:inputVars1})
\item transverse impact parameter $|IP_{2D}| < 0.2$~cm (Figure \ref{fig:inputVars1})
\item transverse momentum $p_t > 1$~GeV/$c$ (Figure \ref{fig:inputVars2})
\item normalized $\chi^2 < 5$ (Figure \ref{fig:inputVars2})
\item longitudinal impact parameter $|IP_{z}|< 17$~cm (Figure \ref{fig:inputVars2})
\item distance to jet axis $ < 0.07$~cm (Figure \ref{fig:inputVars3})
\item decay length $< 5$ (Figure \ref{fig:inputVars3})
\end{itemize}
Figures~\ref{fig:inputVars1} to \ref{fig:inputVars3} show the track selection variables without any cuts, while Figures~\ref{fig:inputVars1N} to \ref{fig:inputVars3N} show the same variables but with all cuts, except for the cut on the variable displayed ($n$-1 cuts).
\begin{figure}[h!]
\centering
\includegraphics[width=0.32\textwidth]{figures/trackNPixelHits_Linear.png}
\includegraphics[width=0.32\textwidth]{figures/trackNHits_Linear.png}
\includegraphics[width=0.32\textwidth]{figures/trackIP2d_Log.png}
\caption{Left: number of hits in the pixel detector, middle: total number of hits in the tracker, right: transverse track impact parameter. No cuts on  track variables were applied.}
\label{fig:inputVars1}
\end{figure}

\begin{figure}[h!]
\centering
\includegraphics[width=0.32\textwidth]{figures/trackTransverseMomentum_Linear.png}
\includegraphics[width=0.32\textwidth]{figures/trackNormChi2_Log.png}
\includegraphics[width=0.32\textwidth]{figures/trackLongitudinalIP_Linear.png}
\caption{Left: transverse track momentum, middle: normalized track $\chi^2$, right: longitudinal impact parameter.  No cuts on track variables were applied.}
\label{fig:inputVars2}
\end{figure}

\begin{figure}[h!]
\centering
\includegraphics[width=0.42\textwidth]{figures/trackDistJetAxis_Linear.png}
\includegraphics[width=0.42\textwidth]{figures/trackDecayLength_Linear.png}
\caption{Left: distance to jet axis, right: track decay length.  No cuts on track variables were applied. }
\label{fig:inputVars3}
\end{figure}

\begin{figure}[h!]
\centering
\includegraphics[width=0.32\textwidth]{figures/trackNPixelHits_cut_Linear.png}
\includegraphics[width=0.32\textwidth]{figures/trackNHits_cut_Linear.png}
\includegraphics[width=0.32\textwidth]{figures/trackIP2d_cut_Log.png}
\caption{Left: number of hits in the pixel detector, middle: total number of hits in the tracker, right: transverse track impact parameter.  All cuts on track selection variables were applied, except for the cut on the displayed quantity.}
\label{fig:inputVars1N}
\end{figure}

\begin{figure}[h!]
\centering
\includegraphics[width=0.32\textwidth]{figures/trackTransverseMomentum_cut_Linear.png}
\includegraphics[width=0.32\textwidth]{figures/trackNormChi2_cut_Log.png}
\includegraphics[width=0.32\textwidth]{figures/trackLongitudinalIP_cut_Linear.png}
\caption{Left: transverse track momentum, middle: normalized track $\chi^2$, right: longitudinal impact parameter.   All cuts on track selection variables were applied, except for the cut on the displayed quantity.}
\label{fig:inputVars2N}
\end{figure}

\begin{figure}[h!]
\centering
\includegraphics[width=0.42\textwidth]{figures/trackDistJetAxis_cut_Linear.png}
\includegraphics[width=0.42\textwidth]{figures/trackDecayLength_cut_Linear.png}
\caption{Left: distance to jet axis, right: track decay length.  All cuts on track selection variables were applied, except for the cut on the displayed quantity. }
\label{fig:inputVars3N}
\end{figure}



This track selection is used for all impact parameter based algorithms, i.e. the track counting and track probability algorithms. The secondary vertex based algorithms apply slightly different selection criteria (only those which are different are listed in the following):
\begin{itemize}
\item jet-track association cone $\Delta R < 0.3$
\item distance to jet axis $ < 0.2$~cm
\item no cut on decay length
\item track quality class = "high purity" 
\end{itemize}

The average number of tracks per jet is displayed in Figure~\ref{fig:trackMult} 
for the case with and without track selection cuts.  The average number of tracks also depends on the jet energy which is shown in Figure~\ref{fig:trackMultVsPt}. The discrepancy between data and simulation is attributed to the Monte Carlo event generator which is not reproducing the charged particle kinematics perfectly.


\begin{figure}[h!]
\centering
\includegraphics[width=0.42\textwidth]{figures/n_tracks_jet_Linear.png}
\includegraphics[width=0.42\textwidth]{figures/n_cutseltracks_jet_Linear.png}
\caption{Left: number of tracks within $\Delta R < 0.5$ of the jet axis. Right: the same for tracks passing the selection criteria of the IP based algorithms as explained in the text.}
\label{fig:trackMult}
\end{figure}

\begin{figure}[h!]
\centering
\includegraphics[width=0.42\textwidth]{figures/track_vs_jetpt_Linear.png}
\includegraphics[width=0.42\textwidth]{figures/cutseltrack_vs_jetpt_Linear.png}
\caption{Left: average number of tracks associated to a jet depending on transverse jet momentum $p_t$. Right:  average number of selected tracks associated to a jet depending on transverse jet momentum $p_t$.}
\label{fig:trackMultVsPt}
\end{figure}

\clearpage
\section{Impact parameter \label{sec:impactparameter}}
The impact parameter (IP) is defined as the minimum distance between
the primary vertex and the trajectory of the track. Tracks produced by
long lived particles such as B mesons are expected to have a sizable
IP.  In the ultra-relativistic limit the IP is Lorentz-invariant to a
good approximation due to the cancellation of boost effects and the
angle of the decay products with respect to the flight path. The
precision of the IP measurement can be different from track to track
and is between 30~$\mu$m and several hundreds $\mu$m. Given that the
uncertainty can be of the same order as the IP value, the IP
significance $S= IP/\sigma_{IP}$ is used for tagging b-jets. 

The IP is ``lifetime signed'': tracks originating from the decay of
particles traveling in the same direction of the jet are signed as
positive, while those in opposite direction are tagged as
negative. This is obtained by using the sign of the scalar product of
the IP segment with the jet direction. It should be noted that a ``sign
flip'' can happen to track produced in the region between the jet
direction and the actual B-hadron flight direction. 


The IP can be measured either in the transverse plane only or in three
dimensions. The high resolution of the pixel detector also along the z
coordinate allows the use of the 3D IP: despite the precision in z being slightly inferior with respect to the one in the
transverse plane, by using the 3D significance the precision is not 
spoiled as the measurement errors are correctly taken into account.   
 

3D Impact Parameter value, error and significance for first, second and
third track in the jet (ordered by IP significance) are displayed in
Figures~\ref{fig:IPfirstTrack} to \ref{fig:IPThirdTrack}. Figure
\ref{fig:IPAllTrack} shows the same for all selected tracks in a jet
(i.e. not ordered by IP significance).

\begin{figure}[h!]
\centering
\includegraphics[width=0.32\textwidth]{figures/IP3d1sorted_Log.png}
\includegraphics[width=0.32\textwidth]{figures/IP3d1Errorsorted_Log.png}
\includegraphics[width=0.32\textwidth]{figures/IP3d1sigsorted_Log.png}
\caption{Left: IP value, middle: IP error, right: IP significance for the first track in the jet, ordered by IP significance.  }
\label{fig:IPfirstTrack}
\end{figure}

\begin{figure}[h!]
\centering
\includegraphics[width=0.32\textwidth]{figures/IP3d2sorted_Log.png}
\includegraphics[width=0.32\textwidth]{figures/IP3d2Errorsorted_Log.png}
\includegraphics[width=0.32\textwidth]{figures/IP3d2sigsorted_Log.png}
\caption{Left: IP value, middle: IP error, right: IP significance for the second track in the jet, ordered by IP significance.  }
\label{fig:IPsecondTrack}
\end{figure}

\begin{figure}[h!]
\centering
\includegraphics[width=0.32\textwidth]{figures/IP3d3sorted_Log.png}
\includegraphics[width=0.32\textwidth]{figures/IP3d3Errorsorted_Log.png}
\includegraphics[width=0.32\textwidth]{figures/IP3d3sigsorted_Log.png}
\caption{Left: IP value, middle: IP error, right: IP significance for the third track in the jet, ordered by IP significance.  }
\label{fig:IPThirdTrack}
\end{figure}

\begin{figure}[h!]
\centering
\includegraphics[width=0.32\textwidth]{figures/trackIP3d_cutsel_Log.png}
\includegraphics[width=0.32\textwidth]{figures/trackIP3dError_cutsel_Log.png}
\includegraphics[width=0.32\textwidth]{figures/trackIP3dsig_cutsel_Log.png}
\caption{Left: IP value, middle: IP error, right: IP significance for all selected tracks in the jet. The track selection as defined in Section~\ref{sec:trackselection} has been applied.}
\label{fig:IPAllTrack}
\end{figure}



The impact parameter has slightly different behaviour depending on the track momentum. This is shown in Figure~\ref{fig:IPbinnedTrackpt} which displays the track IP values for six different track $p_t$ bins.  

% A major source of discrepancies is the number of tracks associated to the jets. This is visible in Figure~\ref{fig:IPBinnedInNtracks} which shows the first track $p_t$ bin split into nince bins of number of tracks per jet.



\begin{figure}[h!]
\centering
\includegraphics[width=0.32\textwidth]{figures/trackIP3d_bin1_cutsel_Log.png}
\includegraphics[width=0.32\textwidth]{figures/trackIP3d_bin2_cutsel_Log.png}
\includegraphics[width=0.32\textwidth]{figures/trackIP3d_bin3_cutsel_Log.png}
\includegraphics[width=0.32\textwidth]{figures/trackIP3d_bin4_cutsel_Log.png}
\includegraphics[width=0.32\textwidth]{figures/trackIP3d_bin5_cutsel_Log.png}
\includegraphics[width=0.32\textwidth]{figures/trackIP3d_bin6_cutsel_Log.png}
\caption{The IP value in six bins of track $p_t$. From top left to bottom right (in units of GeV/$c$): $1<p_t<2; \ 2<p_t<5; \ 5<p_t<8; \ 8<p_t<12; \ 12<p_t<20; \ 20<p_t<50$.   }
\label{fig:IPbinnedTrackpt}
\end{figure}


\clearpage
\section{Secondary vertices \label{sec:secondaryvertex}}
Secondary vertex reconstruction is performed using the “Adaptive Vertex Finder” \cite{bib:AVF}, which performs a fully inclusive vertex search in a list of given tracks. The approach is to fit a vertex from all tracks
and iteratively repeat the fit with tracks that were not compatible with the vertices obtained
in previous iterations. This procedure is repeated until the list of tracks is exhausted or the
vertex fit fails. The “Adaptive Vertex Fitter” is used for the actual vertex fit performed at each
iteration. Since all candidate tracks are passed to it at once, it is able to intrinsically identify and
deal with outliers to allow for the fit to converge. It therefore applies an iterative procedure, by which outlier tracks are increasingly downweighted. This is done until the fit converges and
thus only compatible tracks, which have sizable weights, remain.

The parameters used are primcut = 1.8 and seccut = 6.0. Both denote the track-vertex compatibility
cutoff parameter used for the first and all subsequent fits, respectively. The first fit
attempt is additionally constrained to the beam spot in order to avoid a successful fit of a secondary
vertex with the small cutoff parameter designed for identification of tracks from the
primary vertex. The cutoff parameter of 6.0 is deliberately chosen this large in order to increase
the vertexing efficiency for cases of a b-c decay chain where the two individual secondary and
tertiary vertices cannot be resolved, but both decays yield enough tracks to form a common
vertex. While this vertex definition is slightly unphysical, it increases the b-tagging performance
of the “combined secondary vertex” algorithm. The vertex finder considers a track to
be an outlier if it has been assigned a fit weight of less than 0.5.

The resulting list of vertices is then subject to a cleaning procedure which applies the following
selection criteria:
\begin{itemize}
\item fraction of tracks shared with primary vertex $< 0.65$
\item distance from beam spot in transverse plane $< 2.5$ cm
\item DR of the flight direction with respect to the jet axis $< 0.5$
\item $D_{xy}/\sigma_{D_{xy}}$ (2D “flight distance” significance with respect to reconstructed primary
vertex) $> 3$
\item $D_{xy} > 0.1$ mm
\end{itemize}
In addition, a rejection of vertices due to $K_s$ mesons is applied by rejecting vertices with an invariant mass in the $K_s$ mass window of $0.5 \pm 0.05$~GeV$/c^2$. 


The average charged track multiplicity of a B hadron decay is about five and despite the tight
track quality cuts the efficiency of being able to reconstruct respective decay vertices is very
high. Efficiency limiting factors in reconstruction arise from tracking inefficiencies, tracks lost
due to quality or acceptance cuts or tracks that are also compatible with the primary vertex and
hence excluded from the secondary vertex fit. The number of reconstructed vertices per jet is shown on the left in Figure \ref{fig:vertexNtracks}, while  the number of tracks at the reconstructed secondary vertex is shown in the middle. The dependence of the track multiplicity on the jet momentum is shown on the right. It is visible that the fraction ob b-jets is significantly enhanced for vertices with three or more tracks. This is also visible in  Figure \ref{fig:vertexMass} which shows the reconstructed vertex mass with a minimum of two (left plot) or three (middle plot) tracks at the Secondary Vertex. The right plot in Figure \ref{fig:vertexMass} shows the transverse momentum of the secondary vertex (with two tracks) which is determined using the sum of the momentum vectors of the  tracks at the vertex.

\begin{figure}[h!]
\centering
\includegraphics[width=0.32\textwidth]{figures/sv_nvertices0_Log.png}
\includegraphics[width=0.32\textwidth]{figures/sv_trackmul_Linear.png}
\includegraphics[width=0.32\textwidth]{figures/sv_track_vs_jetpt_Linear.png}
\caption{Left: number of reconstructed secondary vertices per jet, middle: number of tracks at the reconstructed secondary vertex, right: average number of tracks at the secondary vertex versus jet $p_t$. }
\label{fig:vertexNtracks}
\end{figure}

\begin{figure}[h!]
\centering
\includegraphics[width=0.32\textwidth]{figures/sv_mass_Linear.png}
\includegraphics[width=0.32\textwidth]{figures/sv_mass_3tr_Linear.png}
\includegraphics[width=0.32\textwidth]{figures/sv_vtx_pt_Linear.png}
\caption{Left: vertex mass with two or more reconstructed tracks at the vertex. Middle: vertex mass with three or more tracks at the vertex. Right: transverse momentum of the secondary vertex (with two or more tracks). }
\label{fig:vertexMass}
\end{figure}


An important quantity which is sensitive to the lifetime of B hadron decays is the  distance between primary and secondary vertex. As for the impact paramter, the significance of this quantity is used in b-tagging algorithms. Figure \ref{fig:vertexdistance} shows the flight distance significance, the normalized $\chi^2$ of the vertex fit and the energy ratio of tracks at the secondary vertex with respect to all tracks in the jet.

\begin{figure}[h!]
\centering
\includegraphics[width=0.32\textwidth]{figures/sv_flightsig3d_Log.png}
\includegraphics[width=0.32\textwidth]{figures/sv_eratio_Linear.png}
\includegraphics[width=0.32\textwidth]{figures/sv_normchi2_Linear.png}
\caption{Left: vertex flight distance significance. Middle: ratio of track energy at the secondary vertex with respect to all selected tracks in the jet. Right: vertex fit normalized $\chi^2$. }
\label{fig:vertexdistance}
\end{figure}

Two different directions can be defined at the secondary vertex: the flight direction, which points from the primary vertex to the secondary vertex and the direction of the vertex momentum which is the sum of all vertex track momenta. The angle between these two directions measured in $\Delta R$ as well as the angle with the jet axis are shown in Figure~\ref{fig:vertexAngles}.

\begin{figure}[h!]
\centering
\includegraphics[width=0.32\textwidth]{figures/sv_deltar_jet_Linear.png}
\includegraphics[width=0.32\textwidth]{figures/sv_deltar_sum_jet_Linear.png}
\includegraphics[width=0.32\textwidth]{figures/sv_deltar_sum_dir_Linear.png}
\caption{Left: angular distance in $\Delta R$ between jet axis and vertex direction. Middle: angular distance in $\Delta R$ between jet axis and the sum of track momenta at the vertex, Right: angular distance in $\Delta R$ between vertex direction and the sum of track momenta at the vertex. }
\label{fig:vertexAngles}
\end{figure}
\clearpage
\section{Discriminators and tagging rates \label{sec:discriminators}}
Details about the calculation of the discriminators are given in~\cite{btaggingPAS2009}. The distributions of track counting discriminators are shown in Figure~\ref{fig:trackCountingDisc}, jet probability discriminators are shown in Figure~\ref{fig:JetProbDisc} and simple secondary vertex discriminators are shown in Figure~\ref{fig:SimpleSVDisc}. 

\begin{figure}[h!]
\centering
\includegraphics[width=0.45\textwidth]{figures/discri_tche_Log.png}
\includegraphics[width=0.45\textwidth]{figures/discri_tchp_Log.png}
\caption{Left: track counting high efficiency, Right: track counting high purity discriminators. }
\label{fig:trackCountingDisc}
\end{figure}
\begin{figure}[h!]
\centering
\includegraphics[width=0.45\textwidth]{figures/discri_jetprob_Log.png}
\includegraphics[width=0.45\textwidth]{figures/discri_jetbprob_Log.png}
\caption{Left: jet probability , Right: jet B probability discriminators. }
\label{fig:JetProbDisc}
\end{figure}
\begin{figure}[h!]
\centering
\includegraphics[width=0.45\textwidth]{figures/discri_ssche_Linear.png}
\includegraphics[width=0.45\textwidth]{figures/discri_sschp_Linear.png}
\caption{Left: simple secondary vertex high efficiency , Right: simple secondary vertex high purity discriminators. The underflow bin for jets which do not contain a reconstructed secondary vertex is not displayed.}
\label{fig:SimpleSVDisc}
\end{figure}

The tagging rates can be calculated by integrating the discriminator distributions (Figures \ref{fig:trackCountingDisc} to \ref{fig:SimpleSVDisc}) from a given discriminator cut to infinity, divided by the total integral.  The tagging rates are displayed in Figures~\ref{fig:trackCountingDiscEff} to \ref{fig:SimpleSVDiscEff}.

\begin{figure}[h!]
\centering
\includegraphics[width=0.45\textwidth]{figures/tagRate_discri_tche_Log.png}
\includegraphics[width=0.45\textwidth]{figures/tagRate_discri_tchp_Log.png}
\caption{Left: track counting high efficiency tagging rate, Right: track counting high purity tagging rate. }
\label{fig:trackCountingDiscEff}
\end{figure}
\begin{figure}[h!]
\centering
\includegraphics[width=0.45\textwidth]{figures/tagRate_discri_jetprob_Log.png}
\includegraphics[width=0.45\textwidth]{figures/tagRate_discri_jetbprob_Log.png}
\caption{Left: jet probability tagging rate, Right: jet B probability tagging rate. }
\label{fig:JetProbDiscEff}
\end{figure}
\begin{figure}[h!]
\centering
\includegraphics[width=0.45\textwidth]{figures/tagRate_discri_ssche_Linear.png}
\includegraphics[width=0.45\textwidth]{figures/tagRate_discri_sschp_Linear.png}
\caption{Left: simple secondary vertex high efficiency tagging rate, Right: simple secondary vertex high purity tagging rate. }
\label{fig:SimpleSVDiscEff}
\end{figure}
\input{jetswithmuons.tex}
\cleardoublepage
\section{Alignment \label{sec:alignment}}

%%%%%%%%%%%%%%%%%%%%%%%%%%%%%%%%%%%%%%%%%%%%%%%%%%%%%%%%%%%%%%%%%%%%
\subsection{Alignment of the Inner Silicon Tracker}
%%%%%%%%%%%%%%%%%%%%%%%%%%%%%%%%%%%%%%%%%%%%%%%%%%%%%%%%%%%%%%%%%%%%
The determination of the alignment corrections for the Inner Silicon
Tracker to be used for the reprocessing of the 2010 data was done with
the global method~\cite{TkAl_VBlobel,TkAl_Millepede} using a mixture of tracks coming
from atmospheric cosmic rays and minimum bias collisions. 
The global method aligns the highest level structures (half-barrels, endcaps) with all
the six degrees of freedom together with all module units with the most sensitive 
degrees of freedom each: $u$, $w$ and $\gamma$ for the strip modules (also $\alpha$ and $\beta$ in TIB)
and $u$, $v$, $w$ and $\gamma$ for the pixel modules~\footnote{
A local right-handed coordinate system is defined for each module, with $u$ being the
more precisley measured coordinate, $v$ orhogonal to the $u$-axis and in the 
module plane, and the $w$-axis normal to the module plane. Angles $\alpha$, $\beta$ and $\gamma$ are the rotations
around the $u$, $v$ and $w$-axes respectively.}.

During the 2010 LHC run, the geometry of the pixels was monitored on a
daily basis  using the so-called unbiased track-to-primary vertex
residuals. This method consists in selectively
removing a track from the event, determine a primary vertex with the
others, compute the transverse and longitudinal projections of impact
parameter of the probe track with respect the primary vertex and
finally studying the mean value of the distribution of the residuals
as a function of $\phi$, $\eta$ of the track.

Despite starting from a pre-calibrated detector, a discontinuity in
the distribution of the mean longitudinal impact parameter as a
function of azimuthal angle was observed, with the height of the step changing
few times during the year. 
This behavior was interpreted as relative movements, up to 90 $\mu$m in
magnitude, along the $z$-direction of the two BPIX half-barrels,
movements which are allowed since the two halves are mechanically independent.

To cope with this, in a common fit separate sets of alignment corrections for each of
the three BPIX layers in each half-barrel and each of the four FPIX
half-disks in each endcap were provided for seven different periods of
the data-taking. The relative position of the modules with
respect the supporting structure was instead assumed to be the same along all
the 2010 data-taking. 
Figure~\ref{fig:TkAl_PV} shows the distribution of longitudinal separation of the BPIX 
half-barrels as estimated from the unbiased track-to-primary vertex residuals 
as a function of time before and after the alignment procedure. 
 
The statistical precision reached by the alignment was checked looking at the
distribution of the median of the unbiased track-to-hit residuals
computed for each module. 
%In the pixel subsystem, the most relevant component for b-tagging, 
The RMS of these distributions amount to 2 $\mu$m (4 $\mu$m) for the r-$\phi$ ($z$) measurement
coordinate in the BPIX and to 6 $\mu$m (11 $\mu$m) for the r-$\phi$
($r$) measurement in the FPIX, for all the data-taking intervals described above.

A set of alignment parameter errors, calibrated to provide pulls of
the track-to-hit residuals with gaussian standard deviation close to
unity, was provided together with the alignment
constants.

Finally, prior to the restart of the 2011 LHC operations, the alignment
corrections for the two BPIX half-barrels (6 dofs) and for the four FPIX half-disks
(3 translational dofs) were recomputed using a sample of about 15k cosmic ray
tracks.
The analysis of the track-to-primary vertex residuals on events from
2011 collisions indicates that a potentially uncorrected separation of
the two BPIX half-barrels is at most 10 $\mu$m.


%%%%%%%%%%%%%%%%%%%%%%%%%%%%%%%%%%%%%%%%%%%%%%%%%%%%%%%%%%%%%%%%%%%%
\subsection{Simulation-based study of the impact of misalignment on
  b-tagging performances}
%%%%%%%%%%%%%%%%%%%%%%%%%%%%%%%%%%%%%%%%%%%%%%%%%%%%%%%%%%%%%%%%%%%%

The performance of the alignment procedure described above is
essentially not limited by the statistical precision.
To properly describe in the simulation the alignment
accuracy reached at the end of 2010, a misalignment scenario was
prepared following the same approach described in ~\cite{TkAl_CRAFT08}
and using a sample of simulated events with approximately the
same composition of the sample used for the alignment in the data. 
The same alignment parameter errors determined in the data
complemented the misalignment scenario, hereafter referred to as MC2010.

The performances of the different b-tagging algorithms obtained with
the MC2010 misalignment scenario were compared with respect those
obtained with a perfectly aligned detector and with a misalignment
scenario, prepared before the installation of the Inner Silicon Tracker
in CMS, supposed at that epoch to describe the uncertainty on the
alignment parameters expected after 10/pb of collected data~\cite{CMS_NOTE_2008-29}.
The results, obtained from a sample of about 1.5 millions simulated
$t\bar{t}$ events are shown in Figure~\ref{fig:TkAl_effpur_newstartup_vs_ideal}.
For all the taggers the performance with the MC2010 scenario are equal
to those obtained with a perfectly aligned detector.

The same sample of simulated events was used to evaluate the deterioration of the
performances of the b-tagging algorithms in case shifts along the
$z$-direction of the two BPIX half-barrels, similar to those observed
in 2010, were present but not corrected by the alignment procedure.
For this study, the positions and orientations of all the other components of the Tracker 
were supposed to be perfectly known.

Three different scenarios  were investigated
corresponding to 40 $\mu$m, 80 $\mu$m and 160 $\mu$m absolute separation of the
BPIX half-barrels. 
No change in the b-tagging performance is observed for the SSVHE
and SSVHP taggers. For the other taggers
(Figure~\ref{fig:TkAl_effpur_zshifts_vs_ideal}), a significant decrease
of the b-tagging efficiency  is observed only for the 160 $\mu$m separation.
For the track-counting taggers the decrease is more pronounced for 
TCHP, about -5.5\% decrease, already setting-in at the medium
contamination working point, while for the TCHE the decrease is visible
only at the loose contamination working point (about -2.5\% reduction).
In case of the CSVB tagger the decrease is about -2.5\% at 160 $\mu$m separation. 
The largest drop in b-tagging efficiency is observed for the JP and JBP taggers, about -7\%   
at all the tight, medium and loose working points
for the 160 $\mu$m separation scenario.

 

\begin{figure}[!h]
  \centering
    \includegraphics[width=0.7\textwidth]{figures/TkAl_PV}
    \caption{Longitudinal separation of the BPIX half-barrels as
      estimated from the unbiased track-to-primary vertex residual
      method as a function of time for the 2010 LHC proton-proton
      run. Empty (filled) dots are the pre-(post-)alignment values.}
    \label{fig:TkAl_PV}
\end{figure}

\begin{figure}[!h]
  \centering
    \includegraphics[width=0.45\linewidth]{figures/TkAl_MC2010_TCHE}
    \includegraphics[width=0.45\linewidth]{figures/TkAl_MC2010_TCHP}
    \includegraphics[width=0.45\linewidth]{figures/TkAl_MC2010_CSV}
    \includegraphics[width=0.45\linewidth]{figures/TkAl_MC2010_JP}
    \caption{Contamination vs. efficiency for the TCHE, TCHP, CSV and JP b-tagging
      algorithms for a scenario describing the estimated current
      accuracy in alignment compared to a perfectly aligned detector.
      For reference the performance expected 
    after 10/pb of collected lumi, based on a previuos study, are also shown.}
    \label{fig:TkAl_effpur_newstartup_vs_ideal}
\end{figure}

\begin{figure}[!h]
  \centering
    \includegraphics[width=0.45\linewidth]{figures/TkAl_BPIXHBDZ_TCHE}
    \includegraphics[width=0.45\linewidth]{figures/TkAl_BPIXHBDZ_TCHP}
    \includegraphics[width=0.45\linewidth]{figures/TkAl_BPIXHBDZ_CSV}
    \includegraphics[width=0.45\linewidth]{figures/TkAl_BPIXHBDZ_JP}
    \caption{Contamination vs. efficiency for the TCHE, TCHP, CSV and JP b-tagging
      algorithms for scenarios with an artificial separation of the two BPIX
      half-barrels of 40, 80, 160 $\mu$m.}
    \label{fig:TkAl_effpur_zshifts_vs_ideal}
\end{figure}


\input{pileup.tex}
\input{newdevelopments.tex}
%
%% **DO NOT REMOVE BIBLIOGRAPHY**
\bibliography{auto_generated}   % will be created by the tdr script.

%% examples of appendices. **DO NOT PUT \end{document} at the end
\clearpage
\appendix
\input{appendixA.tex}

\end{document}

