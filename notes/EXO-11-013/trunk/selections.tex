\section{The Dataset and Trigger}

The CMS is a general purpose detector providing excellent momentum and direction measurements of particles produced in $pp$ collisions of the LHC beams. The central feature of the Compact Muon Solenoid (CMS) apparatus is a superconducting solenoid of 6~m internal diameter. Within the field volume are the silicon pixel and strip tracker, the crystal electromagnetic calorimeter (ECAL) and the brass/scintillator hadron calorimeter (HCAL). Muons are measured in gas-ionization detectors embedded in the steel return yoke. In addition to the barrel and endcap detectors, CMS has extensive forward calorimetry. CMS uses a right-handed coordinate system, with the origin at the nominal interaction point, the $x$-axis pointing to the center of the LHC, the $y$-axis pointing up (perpendicular to the LHC plane), and the $z$-axis along the anticlockwise-beam direction. The polar angle, $\theta$, is measured from the positive $z$-axis and the azimuthal angle, $\phi$, is measured in the $x$-$y$ plane. The pseudorapidity $\eta=- \ln{(\tan{\theta/2})}$ is frequently used instead of the polar angle $\theta$. Here we only briefly describe the components of the CMS directly relevant to this analysis, the full details of the detector, its subsystems and performance is described elsewhere~\cite{CMS}. 

The inner tracker measures charged particles within the pseudorapidity range $|\eta| < 2.5$. It consists of 1440 silicon pixel and 15\,148 silicon strip detector modules and is located in the 3.8~T field of the superconducting solenoid. It provides an impact parameter resolution of $\sim$\,15~$\mu$m and a transverse momentum ($p_{\rm T}$) resolution of about 1.5\,\% for 100~GeV/$c$ particles. The muons are measured in the pseudorapidity window $|\eta|< 2.4$, with detection planes made of three technologies: Drift Tubes, Cathode Strip Chambers, and Resistive Plate Chambers. Matching the muons to the tracks measured in the silicon tracker results in a transverse momentum resolution between 1 and 5\,\%, for $p_{\rm T}$ values up to 1~TeV/$c$. 

Because of the high rate of the collisions, the CMS uses a two-level dedicated trigger system. The first level (L1) of the CMS trigger, composed of custom hardware processors, is designed to select, in less than 1~$\mu$s, the most interesting events, using information from the calorimeters and muon detectors. The High Level Trigger (HLT) processor farm, running a simplified and highly optimized version of the CMS offline reconstruction, further decreases the event rate from up to 100~kHz to 100~Hz, before data storage.

The data used in this analysis have been collected by the CMS experiment during the 2010 run of the LHC at the center-of-mass energy of 7 TeV and corresponds to the luminosity of $35\pm1.4$ pb$^{-1}$. The data have been collected using inclusive non-isolated muon triggers with the lowest available $p_T$ threshold. In Level 1, the data are selected using muon candidates reconstructed by the L1 muon hardware followed by the confirmation at the HLT, where muons are reconstructed by matching standalone muon tracks with the tracks reconstructed in the silicon tracker detectors to refine the muon transverse momentum $p_T$ measurement. Because the trigger configuration was changing during the data taking period, there are three distinct parts of the dataset where the triggers used had transverse momentum thresholds of 9, 11 and 15~GeV/$c$ at the HLT. In all cases, Level 1 thresholds were low enough to ensure that the HLT thresholds are at the plateau of the Level-1 muon efficiency. To make the selected data uniform, we additionally require offline events to contain at least one trigger candidate with $p_T>15$~GeV/$c$ as measured online, such that the final dataset is the same as it would be as though it had been collected using a single inclusive muon trigger with $p_T>15$~GeV/$c$.
