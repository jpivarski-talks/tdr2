\section{Introduction}
\label{sec:intro}

Recent astrophysical observations implying excess of high energy positrons in the cosmic ray spectrum~\cite{Pamela-positron} have motivated the rise of new physics scenarios suggesting that this excess may be associated with the annihilations of the dark matter particles~\cite{Arkani-Hamed}. In these models the annihilation of a TeV-scale dark 
matter in the galactic halo accounts for the anomalous 
excess of cosmic ray leptons and accommodates discrepancies in direct searches for dark 
matter~\cite{Dama}. One realization of such models assumes an extra $U(1)$ gauge symmetry with weak 
coupling to the Standard Model (SM). The $U(1)$ symmetry is broken, leading to a light massive vector boson 
($m \sim O(1$~GeV/$c^2$)), a ``hidden'' or ``dark'' photon, which can have a small kinetic mixing with the SM photon 
providing a portal for the hidden sector photon to decay to leptons and, if kinematically allowed, hadrons. More complex 
models can lead to a whole hierarchy of the dark sector states. Hidden sectors can be naturally realized in 
supersymmetric models  where coupling of the dark sector to the SUSY sector can be enhanced. At the LHC,
if the energy is sufficient for producing SUSY particles, these models predict production of ``dark photons'' as part
of the SUSY cascades. Such dark photons would decay to pairs of leptons and/or hadrons via kinetic mixing
with photons or, in more complex scenarios, can have preferential coupling to leptons~\cite{mediator-lepton-coupling2}. Assuming moderate coupling 
strength of the hidden sector to the SUSY sector, the new light hidden states may be produced in decays of the 
lightest SUSY particle (LSP), e.g. the lightest neutralino as in the case of the Minimal Supersymmetric Model (MSSM) 
extended by the hidden dark sector. The lightest neutralino can decay either entirely to the light hidden sector particles
and sparticles while a heavier dark fermion providing the cold dark matter candidate, or to light hidden particles 
and a dark neutralino~\cite{BaiHan}, which becomes the new cold dark matter candidate. Because the LSP
in these models is no longer stable, there are scenarios~\cite{Ruderman} where the squark is chosen to be 
the LSP, which would decay into the light hidden sector states. Depending on the complexity of the dark sector,
at the LHC one may expect cascade decays of the light hidden states with emission of multiple dark photons 
leading to characteristic ``lepton jet'' signatures.

A similar signature involving production of new light bosons decaying to pairs of leptons  is predicted in the context the 
Next-to-MSSM (NMSSM) models~\cite{Nilles:1982dy,Frere:1983ag,Ellis:1988er,Drees:1988fc,Ellwanger:1993xa,Ellwanger:1995ru,Miller:2003ay}, which extend MSSM by new singlet superfield weakly coupling to the standard MSSM sector. Compared to the MSSM, the NMSSM greatly reduces the fine-tuning and provides an elegant solution for the ``$\mu$-problem''~\cite{mu-problem}. The NMSSM has an expanded higgs sector and provides new decay modes for the SM-like higgs $h_1$ to a pair of the new light CP-odd higgs states, $a_1$, significantly weakening higgs mass limits from direct searches at LEP~\cite{lep2exclusion}. If $m(a_1)<2 m(\tau)$, there is a substantial branching ratio for $a_1 \to \mu \mu$ decays leading to similar ``muon jet'' signatures.

While previous searches~\cite{D01,D02} for lepton jets in the context of SUSY with a dark sector at the Tevatron did not observe signals of the new physics consistent with the topology of lepton jets, the lower center of mass energy may make these signatures inaccessible. Similarly, the sensitivity of the Tevatron search~\cite{Abazov:2009yi} for $h_1 \to a_1 a_1 \to 4 \mu$ is very limited and has a small impact on the allowed NMSSM parameter space~\cite{Belyaev:nmssm}. Due to the high center-of-mass energy, the LHC may provide access to production of these new states with even relatively small amounts of data motivating searches for anomalous production of collimated groups of leptons in the early LHC data. This note describes a signature-based search for production of new light resonances decaying to pairs of muons using the first 35 pb$^{-1}$ of data collected by the CMS experiment during the 2010 LHC run. The results allow a model-independent interpretation as well as set limits on specific benchmark scenarios in the context of SUSY and NMSSM Higgs described in~\cite{Ruderman,BaiHan} and ~\cite{Belyaev:nmssm}, respectively.
