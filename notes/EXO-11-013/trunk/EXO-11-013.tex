
% svn info. These are modified by svn at checkout time.
% The last version of these macros found before the maketitle will be the one on the front page,
% so only the main file is tracked.
% Do not edit by hand!
\RCS$Revision: 37835 $
\RCS$HeadURL: svn+ssh://svn.cern.ch/reps/tdr2/notes/EXO-11-013/trunk/EXO-11-013.tex $
\RCS$Id: EXO-11-013.tex 37835 2011-02-08 16:20:38Z safonov $
%%%%%%%%%%%%% ptdr definitions %%%%%%%%%%%%%%%%%%%%%
\input{ptdr-definitions}
%%%%%%%%%%%%%%%  Title page %%%%%%%%%%%%%%%%%%%%%%%%
\cmsNoteHeader{EXO-11-013} % This is over-written in the CMS environment: useful as preprint no. for export versions
\title{Search For Resonant Production of Lepton Jets}% Force line breaks with \\

%Author is always "The CMS Collaboration" for PAS and papers, so author, etc, below will be ignored in those cases
\address[tamu]{Texas A\&M University}
\author[tamu]{J. Pivaski, A. Safonov, A. Tatarinov}
\author[cern]{The CMS Collaboration}

% please supply the date in yyyy/mm/dd format. Today has been
% redefined to do so, but it should be fixed as of the final release date.
% For papers and PAS, \today is taken as the date the head file (this one) was last modified according to svn: see the RCS Id string above.
\date{\today}

% Abstract processing:
% 1. **DO NOT use \include or \input** to include the abstract: our abstract extractor will not search through other files than this one.
% 2. **DO NOT use %** to comment out sections of the abstract: the extractor will still grab those lines (and they won't be comments any longer!).
% 3. **DO NOT use tex macros** in the abstract: External TeX parsers used on the abstract don't understand them.
\abstract{
A signature-based search for groups of collimated muons (muon jets) is performed using 35 pb$^{-1}$ of
data collected by the CMS experiment at the LHC, at a center-of-mass energy of 7 TeV.  The analysis aims to search for production of new low-mass states decaying into pairs of muons, and is designed to achieve high sensitivity to a broad range of models predicting muon jet signatures. With no excess observed in the data over the background expectation, 
upper limits on the production cross section times branching ratio times acceptance are derived for several event topologies and range from 0.1-0.5~pb. in addition, the results are interpreted for several benchmark models in the context of SUSY with a new light dark sector yielding limits on new physics production exceeding the Tevatron reach.
}

% Do not comment out the following hypersetup lines (metadata). They will disappear in NODRAFT mode and are needed by CDS.
% Also: make sure that the values of the metadata items are sensible. For APS submissions, they are automatically converted to APS keywords.
\hypersetup{%
pdfauthor={George Alverson, Lucas Taylor, A. Cern Person},%
pdftitle={Search For Resonant Production of Lepton-Jets},%
pdfsubject={CMS},%
pdfkeywords={CMS, physics, software, computing}}

\maketitle %maketitle comes after all the front information has been supplied

%%%%%%%%%%%%%%%%%%%%%%%%%%%%%%%%  Begin text %%%%%%%%%%%%%%%%%%%%%%%%%%%%%
%% **DO NOT REMOVE THE BIBLIOGRAPHY** which is located before the appendix

\tracinginput{intro.tex}
\tracinginput{selections.tex}
\tracinginput{mass_analysis.tex}
\tracinginput{conclusions.tex}


%% **DO NOT REMOVE BIBLIOGRAPHY**
\bibliography{auto_generated}   % will be created by the tdr script.
